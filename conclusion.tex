%%%%%%%%%%%%%%%%%%%%%%%%%%%%%%%%%%%%%%%%%%%%%%%%%%%
%
%  New template code for TAMU Theses and Dissertations starting Fall 2012.  
%  For more info about this template or the 
%  TAMU LaTeX User's Group, see http://www.howdy.me/.
%
%  Author: Wendy Lynn Turner 
%	 Version 1.0 
%  Last updated 8/5/2012
%
%%%%%%%%%%%%%%%%%%%%%%%%%%%%%%%%%%%%%%%%%%%%%%%%%%%
%%%%%%%%%%%%%%%%%%%%%%%%%%%%%%%%%%%%%%%%%%%%%%%%%%%%%%%%%%%%%%%%%%%%%%
%%                           SECTION VI
%%%%%%%%%%%%%%%%%%%%%%%%%%%%%%%%%%%%%%%%%%%%%%%%%%%%%%%%%%%%%%%%%%%%%



\chapter{\uppercase{Conclusions and Future Work}}

The paper presents a scalable distributed Seismic Data Analytics toolkit that is implemented on top of Apache Hadoop and Spark big data analytics platforms. It is an attempt to apply HPC optimizations to big seismic data analytics applications, and simplify the parallelism efforts. The experiments and related analysis given in this paper also shows that this toolkit provides promised scalability, by which performance enhancement can be achieved by increasing cluster hardware resources or tuning distribution parameters without refactoring the application code.

In the future, advanced data distribution strategies such as tiling and bricking with 3D overlapping will be implemented, as well as the application utilities and web service for data browsing and analytics purposes will be developed to better facilitate the works of scientists. Both computation and visualization will be put under consideration to design scalable data processing solutions on big data analytics platforms.



%%%%%%%%%%%%%%%%%%%%%%%%%%%%%%%%%%%%%%%%%%%%%%%%%%%%%%%
%\begin{figure}[H]
%\centering
%\includegraphics[scale=.50]{figures/Penguins.jpg}
%\caption{Another TAMU figure}
%\label{fig:tamu-fig4}
%\end{figure}
%%%%%%%%%%%%%%%%%%%%%%%%%%%%%%%%%%%%%%%%%%%%%%%%%%%%%%%


